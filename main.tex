\documentclass{article}
\usepackage[utf8]{inputenc}
\usepackage{indentfirst}
\usepackage[english,main=greek]{babel}
\usepackage{geometry}
 \geometry{
 a4paper,
 total={175mm,265mm},
 left=20mm,
 top=14mm,
 }
\twocolumn

\usepackage[sorting=debug
]{biblatex}
\usepackage{csquotes} % use this to resolve the following warning: "Package biblatex Warning: 'babel/polyglossia' detected but 'csquotes' missing. Loading 'csquotes' recommended."
\addbibresource{bibliography.bib}

\usepackage[breaklinks]{hyperref} % used this to fix url links: https://www.reddit.com/r/LaTeX/comments/9d9u97/some_reference_get_weird_space_between_them_when/

\pagenumbering{gobble}

% for setting urldate in biblatex to today's date
\DeclareSourcemap{
  \maps[datatype=bibtex]{
    \map[overwrite=true]{
      \pertype{online}
      \step[fieldset=urldate,fieldvalue={{\the\year-\the\month-\the\day}}]
    } 
 }
}

\title{\hspace{0.25cm}\textbf{Πως να Διαβάσεις μια Εργασία\footnote{Μετάφραση: Χρήστος Χ. Μηλαροκώστας, Τμήμα Πληροφορικής και Τηλεπικοινωνιών, Εθνικό και Καποδιστριακό Πανεπιστήμιο Αθηνών (ΕΚΠΑ), Ελλάδα \selectlanguage{english}{(e-mail: chmil@di.uoa.gr)}}\\ \selectlanguage{english}{(How to Read a Paper)}\\ \hspace{0.25cm}\normalsize{\selectlanguage{english}{Version of February 17, 2016}}}}
\author{\selectlanguage{english}{S. Keshav}\\ \selectlanguage{english}{David R. Cheriton School of Computer Science, University of Waterloo}\\ \selectlanguage{english}{Waterloo, ON, Canada}\\ \selectlanguage{english}{keshav@uwaterloo.ca}}

\date{}

\begin{document}

\maketitle

\section*{ΠΕΡΙΛΗΨΗ}
\par\noindent
Οι ερευνητές περνούν ένα μεγάλο μέρος του χρόνου τους διαβάζοντας ερευνητικές εργασίες. Ωστόσο, αυτή η δεξιότητα σπανίως διδάσκεται, οδηγώντας σε πολύ σπαταλημένη προσπάθεια. Αυτό το άρθρο σκιαγραφεί μία πρακτική και αποδοτική \textit{μέθοδο τριών-περασμάτων} για την ανάγνωση ερευνητικών εργασιών. Επίσης, περιγράφω πως να χρησιμοποιείτε αυτή τη μέθοδο για να διεξάγετε μία βιβλιογραφική επισκόπηση.

\section{ΕΙΣΑΓΩΓΗ}
\par
Οι ερευνητές πρέπει να διαβάζουν άρθρα για διάφορους λόγους: για να ασκήσουν κριτική για ένα συνέδριο ή ένα μάθημα, για να παραμένουν ενήμεροι στο πεδίο τους, ή για να διεξάγουν μία επισκόπηση της βιβλιογραφίας ενός νέου πεδίου. Ένας τυπικός ερευνητής θα δαπανήσει πιθανότατα εκατοντάδες ώρες κάθε χρόνο διαβάζοντας εργασίες.
\par
Η εκμάθηση του αποδοτικού διαβάσματος μιας εργασίας είναι μία κρίσιμη δεξιότητα αλλά σπανίως διδάσκεται. Συνεπώς, οι αρχάριοι μεταπτυχιακοί φοιτητές πρέπει να μάθουν μόνοι τους, χρησιμοποιώντας την μέθοδο δοκιμής και σφάλματος. Οι φοιτητές σπαταλούν πολλή προσπάθεια στη διαδικασία και συχνά οδηγούνται σε αγανάκτηση.
\par
Για πολλά χρόνια έχω χρησιμοποιήσει μία απλή προσέγγιση \textquoteleft τριών-περασμάτων\textlatin{'} για να με αποτρέψει από το να πνιγώ στις λεπτομέρεις μιας εργασίας προτού πάρω μία πανοραμική ματιά. Μου επιτρέπει να εκτιμήσω πόσος χρόνος απαιτείται για να αναθεωρήσω ένα σύνολο από εργασίες. Επιπλέον, μπορώ να προσαρμόσω το βάθος της εκτίμησης με βάση τις ανάγκες μου και το πόσο χρόνο διαθέτω. Αυτή η εργασία περιγράφει την προσέγγιση και τη χρήση της κατά τη διεξαγωγή μίας βιβλιογραφικής επισκόπησης.

\section{Η ΠΡΟΣΕΓΓΙΣΗ ΤΡΙΩΝ - ΠΕΡΑΣΜΑΤΩΝ}
\par
Η βασική ιδέα είναι ότι θα πρέπει να κάνεις σε μία εργασία μέχρι τρία περάσματα, αντί να ξεκινάς από την αρχή και να συνεχίζεις τη διαδρομή σου μέχρι το τέλος. Κάθε πέρασμα επιτυγχάνει συγκεκριμένους στόχους και χτίζει στο προηγούμενο πέρασμα: Το \textit{πρώτο} πέρασμα σου παρέχει μία γενική ιδέα για την εργασία. Το \textit{δεύτερο} πέρασμα σου επιτρέπει να πιάσεις το περιεχόμενο της εργασίας, αλλά όχι τις λεπτομέρειες. Το \textit{τρίτο} πέρασμα σε βοηθάει να καταλάβεις την εργασία σε βάθος.
\subsection{Το πρώτο πέρασμα}
\par
Το πρώτο πέρασμα είναι ένα γρήγορο σκανάρισμα για την απόκτηση μίας πανοραμικής ματιάς της εργασίας. Επίσης, μπορείς να αποφασίσεις εάν χρειάζεται να προχωρήσεις σε παραπάνω περάσματα. Αυτό το πέρασμα θα πρέπει να πάρει περί των πέντε με δέκα λεπτών και αποτελείται από τα ακόλουθα βήματα:
\begin{enumerate}
    \item Προσεχτικά διάβασε τον τίτλο, την περίληψη και την εισαγωγή
    \item Διάβασε τις επικεφαλίδες των ενοτήτων και υπο-ενοτήτων, αλλά αγνόησε όλα τα άλλα
    \item Ρίξε μια ματιά στο μαθηματικό περιεχόμενο (εάν υπάρχει) για να καθορίσεις τα υποκείμενα θεωρητικά θεμέλια
    \item Διάβασε τα συμπεράσματα
    \item Ρίξε μια ματιά στις παραπομπές, τσεκάροντας στο μυαλό σου αυτές που έχεις ήδη διαβάσει
\end{enumerate}
\par
Στο τέλος του πρώτου περάσματος, θα πρέπει να είσαι σε θέση να απαντήσεις τα \textit{πέντε \textlatin{Cs}}:
\begin{enumerate}
    \item \textit{\textlatin{Category} (Κατηγορία):} Τι τύπος εργασίας είναι\textlatin{;} Μία εργασία μετρήσεων\textlatin{;} Μια ανάλυση ενός υπάρχοντος συστήματος\textlatin{;} Μία περιγραφή ενός ερευνητικού πρωτότυπου\textlatin{;}
    \item \textit{\textlatin{Context} (Περιεχόμενο):} Με ποιες άλλες εργασίες σχετίζεται\textlatin{;} Ποιες θεωρητικές βάσεις χρησιμοποιούνται για την ανάλυση του προβλήματος\textlatin{;}
    \item \textit{\textlatin{Correctness} (Ορθότητα):} Φαίνονται έγκυρες οι υποθέσεις\textlatin{;}
    \item \textit{\textlatin{Contributions} (Συνεισφορές):} Ποιες είναι οι βασικές συνεισφορές της εργασίας\textlatin{;}
    \item \textit{\textlatin{Clarity} (Σαφήνεια):} Είναι καλογραμμένη η εργασία\textlatin{;}
\end{enumerate}
\par
Χρησιμοποιώντας αυτή την πληροφορία, μπορείς να διαλέξεις να μην διαβάσεις περαιτέρω την εργασία (και να μην την τυπώσεις, σώζοντας έτσι δέντρα). Αυτό μπορεί να συμβει γιατί η εργασία δεν σε ενδιαφέρει, ή γιατί δεν γνωρίζεις αρκετά για την περιοχή ώστε να καταλαβείς την εργασία, ή γιατί οι συγγραφείς κάνουν μη έγκυρες υποθέσεις. Το πρώτο πέρασμα είναι επαρκές για εργασίες που δεν εντάσονται στην ερευνητική σου περιοχή, αλλά μπορεί κάποια μερά να καταστούν σχετικές.
\par
Παρεπιπτόντως, όταν γράφεις μία εργασία, μπορείς να αναμένεις ότι οι περισσότεροι κριτικοί (και αναγνώστες) θα κάνουν μόνο ένα πέρασμα. Φρόντισε να επιλέγεις συνεκτικούς τίτλους για τις ενότητες και υπο-ενότητες και να γράφεις περιεκτικές και πλήρεις περιλήψεις. Εάν ένας κριτής δεν μπορεί να καταλάβει την ουσία μετά από ένα πέρασμα, η εργασία πιθανότατα θα απορριφθεί; εάν ένας αναγνώστης δεν μπορεί να καταλάβει τα καλύτερα κομμάτια της εργασίας μετά από πέντε λεπτά, η εργασία πιθανότατα δε θα διαβαστεί ποτέ. Για αυτούς του λόγους, μία \textquoteleft γραφική περίληψη\textlatin{'} η οποία συνοψίζει την εργασία με ένα καλά επιλεγμένο σχήμα είναι μια εξαιρετική ιδέα και συναντάται σε όλο και περισσότερα επιστημονικά περιοδικά.
\subsection{Το δεύτερο πέρασμα}
\par
Στο δεύτερο πέρασμα, διάβασε την εργασία με μεγάλη προσοχή, αλλά αγνόησε λεπτομέρειες όπως οι αποδείξεις. Βοηθάει να σημειώνεις σημεία κλειδιά, ή να κάνεις σχόλια στα περιθώρια, καθώς διαβάζεις. Ο \textlatin{Dominik Grusemann} από το Πανεπιστήμιο του Άουγκσμπουργκ προτείνει να \enquote*{σημειώνεις όρους που δεν καταλαβαίνεις, ή ερωτήσεις που μπορεί να θες να κάνεις στους συγγραφείς}. Εάν έχεις το ρόλο του κριτή μιας εργασίας, αυτά τα σχόλια θα σε βοηθήσουν όταν γράφεις την κριτική σου και θα υποστηρίξουν την κριτική σου κατά τη συνάντηση της επιτροπής.
\begin{enumerate}
    \item Κοίτα προσεκτικά τα σχήματα, τα διαγράμματα και όλες τις υπόλοιπες εικονογραφήσεις στην εργασία. Δώσε ιδιαίτερη προσοχή στα γραφήματα. Είναι σωστές οι ετικέτες των αξόνων\textlatin{;} Απεικονίζονται τα αποτελέσματα με γραμμές σφάλματος, ώστε τα συμπεράσματα να είναι στατιστικά σημαντικά\textlatin{;} Κοινά λάθη όπως αυτά θα ξεχωρίσουν τη βιαστική, πρόχειρη δουλειά από την πραγματικά εξαιρετική.
    \item Θυμήσου να σημειώσεις σχετικές αδιάβαστες παραπομπές για περαιτέρω διάβασμα (αυτός είναι ένας καλός τρόπος για να μάθεις περισσότερα για το υπόβαθρο αυτής της εργασίας).
\end{enumerate}
\par
Το δεύτερο πέρασμα θα πρέπει να πάρει μέχρι μία ώρα για έναν έμπειρο αναγνώστη. Μετά από αυτό το πέρασμα, θα πρέπει να μπορείς να πιάσεις το περιεχόμενο της εργασίας. Θα πρέπει να μπορείς να συνοψίσεις την κεντρική ιδέα της εργασίας, με υποστηρικτικά τεκμήρια, σε κάποιον άλλο. Αυτό το επίπεδο λεπτομέρειας είναι κατάλληλο για μία εργασία για την οποία ενδιαφέρεσαι, αλλά δεν συγκαταλέγεται στην ερευνητική σου ειδίκευση.
\par
Μερικές φορές δεν θα καταλαβαίνεις μία εργασία ακόμα και μετά το τέλος του δεύτερου περάσματος. Αυτό μπορεί να συμβαίνει επειδή το αντικείμενο του θέματος είναι καινούριο σε σένα, με μη οικεία ορολογία και ακρωνύμια. Ή μπορεί οι συγγραφείς να χρησιμοποιούν μία απόδειξη ή μία πειραματική τεχνική που δεν καταλαβαίνεις, με αποτέλεσμα το μεγαλύτερο μέρος της εργασίας να είναι ακατανόητο. Η εργασία μπορεί να είναι κακογραμμένη με ατεκμηρίωτους ισχυρισμούς και πληθώρα μεταγενέστερων αναφορών. Ή απλά μπορεί να είναι αργά τη νύχτα και να είσαι κουρασμένος. Μπορείς τώρα να επιλέξεις: (α) να αφήσεις την εργασία στην άκρη, ελπίζοντας ότι δεν θα χρειαστεί να καταλάβεις το υλικό για να είσαι επιχημένος στην καριέρα σου, (β) να επιστρέψεις στην εργασία αργότερα, ίσως μετά το διάβασμα του υποβάθρου ή (γ) να επιμείνεις και να συνεχίσεις στο τρίτο πέρασμα.
\subsection{Το τρίτο πέρασμα}
\par
Για να καταλάβεις πλήρως την εργασία, ειδικά εάν είσαι κριτής, απαιτείται ένα τρίτο πέρασμα. Το κλειδί στο τρίτο πέρασμα είναι να προσπαθήσεις να \textit{επανα-υλοποιήσεις εικονικά} την εργασία: δηλαδή, κάνοντας τις ίδιες υποθέσεις με τους συγγραφείς, να επανα-δημιουργήσεις τη δουλειά τους. Συγκρίνοντας αυτή την επανα-δημιουργία με την πραγματική εργασία, μπορείς εύκολα να αναγνωρίσεις όχι μόνο τις καινοτομίες της εργασίας, αλλά και τις κρυμμένες της αποτυχίες και υποθέσεις.
\par
Αυτό το πέρασμα απαιτεί μεγάλη προσοχή στην λεπτομέρεια. Πρέπει να μπορείς να αναγνωρίσεις και να αμφισβητήσεις κάθε υπόθεση σε κάθε δήλωση. Επιπλέον, πρέπει να σκεφτείς πώς εσύ θα παρουσίαζες μία συγκεκριμένη ιδέα. Αυτή η σύγκριση του πραγματικού με το εικονικό παρέχει μία ευδιάκριτη γνώση στις τεχνικές απόδειξης και παρουσίασης στην εργασία και μπορείς πιθανότατα να την προσθέσεις αυτή στο ρεπερτόριο των εργαλείων σου. Κατά το πέρασμα αυτό, πρέπει επίσης να σημειώνεις ιδέες για μελλοντική δουλειά.
\par
Αυτό το πέρασμα μπορεί να πάρει πολλές ώρες για αρχάριους και περισσότερο από μία ή δύο ώρες ακόμα και για έναν έμπειρο αναγνώστη. Στο τέλος αυτού του περάσματος, πρέπει να μπορείς να επανακατασκευάσεις ολόκληρη τη δομή της εργασίας από μνήμης, όπως και να μπορείς να αναγνωρίσεις τα δυνατά και τα αδύναμα σημεία. Συγκεκριμένα, πρέπει να μπορείς να εντοπίσεις υπονοούμενες υποθέσεις, απούσες παραπομπές σε συναφή δουλειά και πιθανά θέματα με τις πειραματικές ή αναλυτικές τεχνικές.
\section{ΚΑΝΟΝΤΑΣ ΜΙΑ ΒΙΒΛΙΟΓΡΑΦΙΚΗ ΕΠΙΣΚΟΠΗΣΗ}
\par
Οι δεξιότητες ανάγνωσης εργασιών δοκιμάζονται κατά την διεξαγωγή μίας βιβλιογραφικης επισκόπησης. Εκεί απαιτείται να διαβάσεις δεκάδες εργασίες, πιθανώς σε ένα μη οικείο πεδίο. Ποιες εργασίες πρέπει να διαβάσεις\textlatin{;} Να πως μπορείς να χρησιμοποιήσεις την προσέγγιση τριών-περασμάτων σε βοήθεια σου.
\par
Πρώτα, χρησιμοποιήσε μία ακαδημαϊκή μηχανή αναζήτησης σαν την \textlatin{Google Scholar} ή την \textlatin{CiteSeer} και μερικές καλά διαλεγμένες λέξεις κλειδιά για να βρεις από τρεις μέχρι πέντε \textit{πρόσφατες με-πολλές-αναφορές}\footnote{[ΣτΜ: Εννοούνται εργασίες στις οποιές αναφέρονται πολλές άλλες εργασίες, εργασίες με πολλές ετεροαναφορές.]} εργασίες στην περιοχή. Κάνε ένα πέρασμα την κάθε εργασία για να πάρεις μία διαίσθηση της δουλειάς, μετά διάβασε τις ενότητές τους με τίτλο σχετική δουλειά\footnote{[ΣτΜ: Είθισται στις εργασίες να ύπαρχει μία ενότητα που τιτλοφορείται Σχετική Δουλειά \textlatin{(Related Work)}, στην οποία οι συγγραφείς παραπέμπουν σε άλλες εργασίες που έχουν ασχοληθεί με το πρόβλημα που εξετάζεται, συνήθως από μία διαφορετική σκοπιά, ή με διαφορετικές υποθέσεις, ή με χρήση άλλων τεχνικών.]}. Θα βρεις μία σύντομη ανακεφαλαίωση της πρόσφατης δουλειάς και πιθανώς, εάν είσαι τυχερός, μία κατεύθυνση προς μία πρόσφατη επισκόπηση. Εάν μπορείς να βρεις μία τέτοια επισκόπηση, τελείωσες. Διάβασε την επισκόπηση, συγχαίροντας τον εαυτό σου για την καλή σου τύχη.
\par
Διαφορετικά, στο δεύτερο βήμα, βρες κοινές παραπομπές και επαναλαμβανόμενα ονόματα συγγραφέων στη βιβλιογραφία. Αυτά είναι οι εργασίες κλειδιά και οι ερευνητές κλειδιά στην περιοχή. Κατέβασε τις εργασίες κλειδιά και άσε τες στην άκρη. Έπειτα πήγαινε στους ιστοτόπους των ερευνητών κλειδιά και δες που έχουν δημοσιεύσει πρόσφατα. Αυτό θα σε βοηθήσει να αναγνωρίσεις τα κορυφαία συνέδρια στην περιοχή γιατί οι καλύτεροι ερευνητές συνήθως δημοσιεύουν στα κορυφαία συνέδρια.
\par
Το τρίτο βήμα είναι να πας στον ιστότοπο καθενός από αυτά τα συνέδρια και να κοιτάξεις τα πρόσφατα πρακτικά. Ένα γρήγορο σκανάρισμα συνήθως θα αναγνωρίσει πρόσφατες υψηλής-ποιότητας σχετικές δουλειές. Αυτές οι εργασίες, μαζί με εκείνες που έβαλες στην άκρη πρωτύτερα, αποτελούν την πρώτη έκδοση της επισκόπησής σου. Κάνε δύο περάσματα αυτές τις εργασίες. Αν όλες παραπέμπουν σε μία εργασία κλειδί που δεν βρήκες πρωτύτερα, απέκτησέ τη και διάβασέ τη, επαναλαμβάνοντας όσο χρειάζεται.
\section{ΣΧΕΤΙΚΗ ΔΟΥΛΕΙΑ}
\par
Εάν διαβάζεις αυτή την εργασία για να κάνεις μία κριτική, πρέπει επίσης να διαβάσεις την εργασία του \textlatin{Timothy Roscoe} με τίτλο \textlatin{\enquote*{Writing reviews for systems conferences}} \cite{3writingReviewsForSystemsConferences}. Εάν σκοπεύεις να γράψεις μία τεχνική εργασία, πρέπει να ανατρέξεις στον πλήρη ιστότοπο του \textlatin{Henning Schulzrinne} \cite{4writingTechnicalArticles} και στην εξαιρετική σύνοψη της διαδικασίας του \textlatin{George Whitesides} \cite{5whitesidesGroupWritingAPaper}. Τέλος, ο \textlatin{Simon Peyton Jones} έχει έναν ιστότοπο που καλύπτει όλο το φάσμα των ερευνητικών δεξιοτήτων \cite{2researchSkills}.
\par
Ο \textlatin{Iain H. McLean} της \textlatin{Psychology, Inc.} έχει συνθέσει έναν \textquoteleft πίνακα κριτικής\textlatin{'} που μπορείς να κατεβάσεις που απλοποποιεί την κριτική εργασιών χρησιμοποιώντας την προσέγγιση τριών-περασμάτων για εργασιές στην πειραματική ψυχολογία \cite{1literatureReviewMatrix}, ο οποίος μπορεί πιθανότατα να χρησιμοποιηθεί, με μικρές μετατροπές, για εργασίες άλλων περιοχών.
\section{ΕΥΧΑΡΙΣΤΙΕΣ}
\par
Η πρώτη έκδοση αυτού του εγγράφου συντάχθηκε από τους φοιτητές μου: \textlatin{Hossein Falaki}, \textlatin{Earl Oliver} και \textlatin{Sumair Ur Rahman}. Οι ευχαριστίες μου σε αυτούς. Επίσης επωφελήθηκα από τα διορατικά σχόλια του \textlatin{Christophe Diot} και από την εξονυχιστική επεξεργασία της \textlatin{Nicole Keshav}.
\par
Θα ήθελα να κάνω αυτό το έγγραφο ζωντανό, ανανεώνοντάς το καθώς λαμβάνω σχόλια. Παρακαλώ πάρτε το χρόνο για να μου στείλετε ένα \textlatin{e-mail} με οποιαδήποτε σχόλια ή προτάσεις για βελτίωση. Ευχαριστώ για την ενθαρρυντική ανάδραση από πολλούς ανταποκριτές ανά τα χρόνια.

\section*{ΑΝΑΦΟΡΕΣ}

\selectlanguage{english}{
\printbibliography[heading=none]
}
\end{document}
